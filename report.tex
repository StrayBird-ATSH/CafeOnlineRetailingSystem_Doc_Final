\documentclass[a4paper]{report}
\pagestyle{headings}
\usepackage{hyperref}
\usepackage{listings}
\usepackage{graphicx}
\usepackage{subfiles}
\usepackage{multirow}
\lstset{numbers=right}
\lstset{breaklines}
\title{Lab Report for Software Engineering course \newline
 Lab 4: Starbubucks coffee online retailing system v3.0}
\author{Wang, Chen\qquad Liu, Jiaxing\qquad Huang, Jiani\qquad Tang, Xinyue \\
16307110064\qquad17302010049\qquad 17302010063\qquad 16307110476 \\
School of Software\\
Fudan University
}
\date{\today}
\bibliographystyle{plain}
\begin{document}
\maketitle

\tableofcontents
\chapter{Overview of this lab}
% Wang, Chen's part below
\section{The Objectives of the Project}
\section{Specifications of the Lab}
\section{The division of work in the team}

\subsection{Division of work: Wang, Chen (Git username: Wang, Chen 16307110064)}
He constructs the overall structure of the project, divides the entire workload into several parts so that each part can be finish the work separately. In addition, he draws the diagram of the entire project on the Huawei cloud platform that contains the parts like Epic, feature, story and tasks. Furthermore, he scratches the outline of how to implement the methods adopted in this project. At last, he summarized the general parts in the documentation and drafted some regulations for commit messages.

\subsection{Division: Huang, Jiani (Git username: Currycurrycurry 17302010063)}
She creates the concrete classes for the diverse drinks and ingredients.

\subsection{Division: Tang, Xinyue (Git username: xinyuetang 16307110476)}
She implements the methods related to order processing, discount processing and total price processing.

\subsection{Division: Liu, Jiaxing (Git username: jiaxingliu 17302010049) }
He tests the implementation in this project via the interface given by the teaching assistants.

% Wang, Chen's part below
\chapter{Improved skills on team collaboration}
\section{Consistent git commit message styles}
\subsection{Commit message requirements}
The following are the requirements for the commit message in our team, this version of specifications are revised according to the commit message style recommendation from website \emph{Commit message guidelines · GitHub}\footnote{\url{https://gist.github.com/robertpainsi/b632364184e70900af4ab688decf6f53}}, \emph{How to Write a Git Commit Message}\footnote{\url{https://chris.beams.io/posts/git-commit/}}, \emph{How To Write a Good Commit Message}\footnote{\url{https://api.coala.io/en/latest/Developers/Writing_Good_Commits.html}} and the git commit message recommendation from one of the most authoritive open source project \textbf{GNOME} \emph{Guidelines for Commit Messages}\footnote{\url{https://wiki.gnome.org/Git/CommitMessages}}.
\begin{enumerate}
\item
Only ASCII characters are allowed in the entire commit message
\item
All commit messages must start with one of the types identified in the following table, all words are lowercase
\item
It is best to have an associated work item, associated with the work item, followed by type, space \# number space followed by content, such as fix \#123 content
\item
The total number of characters recommended in the subject \emph{(note that it is the number of chars instead of the number of words)} is less than 50, and the maximum number is not more than 74 characters (including the previous type and item number, etc.)
\item
There is no need to add a period at the end of the head
\item 
After the type in the subject line, the first letter of the first word after the task number (if any) is capitalized and that indicates the beginning of a sentence
\item
Use the imperative tone in the subject sentence (although it is you who have actually done the work)
\item
The tense of the subject is the general present tense
\item
It is recommended that for commits involving complex modifications, body should be added in addition to the subject for further explanation. The method is as follows: break a new line and then write is the body, [do not follow the head without line break]
\item
For the body part of the commit message, there is no requirement other than writing Chinese, and it is relatively free. It is also recommended to write more than one line instead of one line in order to facilitate reading.
\end{enumerate}

\subsection{Types allowed in the subject of the commit message}
\newcommand{\tabincell}[2]{\begin{tabular}{@{}#1@{}}#2\end{tabular}}
\begin{table}[htbp]
\centering
\caption{\label{tab:test}Commit message types}
\begin{tabular}{|l|l|}
\hline
\textbf{Type} & \textbf{Description} \\ \hline
feat&A new feature\\ \hline
fix&A bug fix\\ \hline
wip&While working on a fix/feature\\ \hline
docs&Documentation only changes\\ \hline
style&\tabincell{c}{Changes that do not affect the meaning of the code \\(white-space, formatting, missing semi-colons, etc) }\\ \hline
refactor&A code change that neither fixes a bug or adds a feature\\ \hline
test&Adding missing tests \\ \hline
chore&\tabincell{c}{Changes to the build process or auxiliary tools \\and libraries such as documentation generation}\\ \hline
\end{tabular}
\end{table}

\chapter{Structure of the project}
% Wang, Chen's part below
% 
% 
% Wang, Chen's part above
\chapter{Implementation of the features}
\section{Entities implementation}
% Huang, Jiani's part below
The implementation of entity package can be divided into two packages: the package of drinkEntity and the one of ingredients.
\par To our attention, I apply this.getClass.getName() to simplify the constructor by avoiding inputting string every time.
\subsection{drinkEntity}
\par At present we have four different kinds of drinks: Cappuccino,Espresso,GreenTea and RedTea. But considering the possibilities of adding other kinds of coffee and tea and the rationality of logic, the abstract classes of Coffee and Tea are applied to construct the whole inheritance relationship.That is, Both Coffee and Tea will inherit the OrderItem class, and all specific kinds of coffee and tea will inherit the abstract classes of Coffee and Tea.
\par We should pay attention to the constructor method of the different concrete classes. Since the initialization only need the different basic price of drinks, we only need to call the setPrice() method in constructor method. And all the generic method of these classes will be implemented in the parent class, the OrderItem class, such as cost() and size2price() methods.
Thanks to the inheritance tree, in the size2price() method, we can use the instanceof keyword to simplify the judge of cup-size.
\subsection{ingredientEntity}
It is evident that the four concrete ingredients inherit the Ingredient class in dto package: chocolate, cream, milk and sugar. The constructor like the above drinkEntity's only need to set the price. 
\subsection{OrderItem}
 I add two methods: size2price and cost in this class in order to make all its subclasses use this method.



% Huang, Jiani's part above
\section{Order services implementation}
% Tang, Xinyue's part below
% 
% 
% Tang, Xinyue's part above
\chapter{Testing of the features}
\section{Testing methods adopted}
% Liu, Jiaxing's part below
% 
% 
% 
% Liu, Jiaxing's part above
\section{Testing results}
% Liu, Jiaxing's part below
% 
% 
% 
% Liu, Jiaxing's part above
\begin{thebibliography}{A}

\bibitem{1}
Wikipedia contributors. (2019, March 22). JUnit. In \emph{Wikipedia, The Free Encyclopedia}. Retrieved 14:53, April 1, 2019, from \url{https://en.wikipedia.org/w/index.php?title=JUnit&oldid=888928403}

\end{thebibliography}
\end{document} 